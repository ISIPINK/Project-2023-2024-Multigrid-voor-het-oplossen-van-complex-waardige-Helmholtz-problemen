
\documentclass[a4paper,12pt]{article}

\setlength{\textwidth}{15.0cm}
\setlength{\textheight}{24.0cm}
\setlength{\topmargin}{0cm}
\setlength{\headsep}{0cm}
\setlength{\headheight}{0cm}
\pagestyle{plain}

\usepackage{hyperref}
\hypersetup{
    colorlinks=true,
    linkcolor=blue,
    filecolor=magenta,      
    urlcolor=blue,
    citecolor=blue,
    linktoc=page
}
\usepackage[dvips]{epsfig}
\usepackage{tikz}
\usepackage[english]{babel}
\usepackage{caption}
\captionsetup{font=it}
\usepackage[autostyle, english=american]{csquotes}

\usepackage[
backend=biber,
style=alphabetic,
]{biblatex}

\renewcommand{\bibfont}{\footnotesize}

\usepackage{amsmath,amssymb,amsthm}

\usepackage{comment}
\usepackage{listings}

% \addbibresource{bibliography2.bib} 
\setlength{\parindent}{0pt}

\selectlanguage{english}
\begin{document}

\title{Multigrid for solving complex-valued Helmholtz problems}
\author{Isidoor Pinillo Esquivel}
\date{\today}
\maketitle

\section{Failure of the Multigrid method for Helmholtz problems: analysis}

\subsection{Discretization}
(a)
\begin{align*}
    10                    & \leq \lambda \text{ \#gridpoints} \Leftrightarrow                \\
    10                    & \leq \frac{2\pi}{\sqrt{|\sigma|}} \frac{1}{h^{d}}\Leftrightarrow \\
    \sqrt{|\sigma|} h^{d} & \leq \frac{2 \pi}{10} \approx 0.625.
\end{align*}

(b)
\[
    \text{\# roosterpunten} = \frac{10 \sqrt{600}}{2 \pi}
    .\]

\subsection{1D model problem}
(a) \\
To proof : $H^{2 h} \neq I_h^{2 h} H^h I_{2 h}^h$.
\begin{align*}
    H^{2h} & = H_n = A_n +\sigma {id}_n          \\
    H^{h}  & = H_{2n} = A_{2n} +\sigma {id}_{2n}
\end{align*}
Assume that $A_n =I_h^{2 h} A_{2n} I_{2 h}^h= R_{2n} A_{2n} I_n$. By linearity it is  sufficient
to proof:
\begin{align*}
    \sigma {id}_n  & \neq \sigma R_{2n} {id}_{2n} I_{n} \Leftrightarrow \\
    {id}_n         & \neq R_{2n} I_{n} \Leftarrow                       \\
    (id_n)_{00} =1 & \neq \frac{3}{4} = (R_{2n} I_{n})_{00}
\end{align*}

First equivalence follows from $\sigma \neq 0$. The assumption and the last inequality depends on the definition of
restriction and interpolation. \\
(b) \\
See code/main.ipynb for code and plots. We implemented $f(t) = \delta(t-0.5)$ by concentrating all the mass into the
middle element of $f_n$.\\
(c) \\
There exists a closed formula for eigenvalues and eigenvectors of tridiagonal toeplitz matrix. It is just tedious to
use. Alternatively the eigenvalues and eigenvectors can be derived from the Poisson problem ($\sigma=0$) because
\begin{align*}
    Av             & = \lambda v \Rightarrow \\
    (A+\sigma id)v & = Av + \sigma v         \\
                   & = (\lambda+\sigma)v
\end{align*}
i.e. eigenvectors stay the same and eigenvalues get shifted by $\sigma$. \\

(d) \\
See code/main.ipynb for the plot. $\sigma = 0$ is the boundary where $H$ goes from indefinite to definite.

\subsection{LFA analysis of the $\omega$-Jacobi smoother}
(a) \\
For grid points without a neighboring boundary point there
$H$ acts like following stencil:
\[
    H_n = n^{2} [ -1 \quad 2+ \frac{\sigma}{n^{2}}  \quad -1]
    .\]

So $R_{\omega}$ works element wise the following way on the error:
\begin{equation}
    e_{j}^{m+1} = (1-\omega) e_{j} ^{m} + \frac{\omega n^{2}}{2 n^{2} + \sigma} (e_{j-1} ^{m} + e_{j+1}^{m})
    .
\end{equation}
Very similar to the analysis for the Poisson equation. Note that we haven't used that $\sigma$ is real.
Doing the LFA subsitution $e_j^{(m)}=\mathcal{A}(m) e^{i j \theta}$:

\begin{align*}
    A(m+1) & = A(m) \left( 1-\omega + \frac{\omega n^{2}}{2 n^{2} + \sigma} (e^{-i\theta} + e^{i\theta}) \right) \\
           & = A(m) \left( 1-\omega + 2 \cos(\theta) \frac{\omega n^{2}}{2 n^{2} + \sigma}   \right)
\end{align*}
The factor in behind of $A(m)$ is the amplification factor $G(\theta)$.

(b) \\
$\sigma = 0$ reduces back to the LFA we did for the Poisson equation.
$\theta \approx 0 \Rightarrow \cos(\theta)
    \approx 1 + O \left(\frac{1}{n^{2}}\right) \Rightarrow G(\theta) \approx 1 - O \left(\frac{1}{n^{2}}\right)$
so smooth modes are preserved for big $n$.
(c) \\
See code/main.ipynb for the plot. \\

(d)\\
Maximum of $G(\theta)$ is achieved at $\theta =0$ because $G(\theta)$ is just
an increasing function of $\cos(\theta)$. This means that
$\rho = 1-\omega + 2 \frac{\omega n^{2}}{2 n^{2} + \sigma}  \approx 1.05 $
which suggests that weighted jacobi wouldn't converge.


\subsection{Spectral analysis of the two-grid correction scheme}
(a) \\
It is easily seen that
\begin{equation}
    R_{2n} =  c S_{2n} (3id -A_{2n}).
\end{equation}
with $c \in \mathbb{R}_{0}$, $S_{2n}: \mathbb{R}^{2n-1} \rightarrow \mathbb{R}^{n-1}: (v_{j})_{j \leq 2n-2}
    \rightarrow (v_{2j+1})_{j \leq n-2}$  subsampling uneven components.
Using that $S_{2n}w_{k}^{2n} = w_{k}^{n}$ if $k<n$ it is easily seen that:
\begin{align*}
    R_{2n} w_{k}^{2n} & = cS_{2n}(3id- A_{2n}) w_{k}^{2n}             \\
                      & = c(3- \lambda_{k}(A_{2n})) S_{2n} w_{k}^{2n} \\
                      & = a(n,k) w_{k}^{n} .
\end{align*}
In our case $I_{n}$ is linear interpolation, reconstuction error for lagrange interpolation can be bounded
using Taylors theorem.
\begin{equation}
    I_{n} S_{2n} v_{2n}  \approx c v_{2n}.
    .
\end{equation}
For $w_{k}^{2n}$ smooth and combining with previous argument we have:
\begin{equation}
    I_{n} R_{2n} w_{2n}  \approx c_{1} w_{2n}.
\end{equation}
To check the normalizing constant try $v_{2n} = 1 \Rightarrow c_{1} =1$. \\
We also checked these facts numerically by plotting see code. Now doing spectral analysis of TG is
straight forward:
\begin{align*}
    TG w_{k}^{2n} & = (id - I_{n} H_{n}^{-1} R_{2n} H_{2n}) w_{k}^{2n}                                    \\
                  & =  w_{k}^{2n} - I_{n} H_{n}^{-1} R_{2n} \lambda_{k}(H_{2n}) w_{k}^{2n}                \\
                  & =  w_{k}^{2n} - I_{n} H_{n}^{-1}  a(n,k) w_{k}^{n} \lambda_{k}(H_{2n})                \\
                  & =  w_{k}^{2n} - I_{n}  a(n,k) w_{k}^{n} \lambda_{k}(H_{n}^{-1})  \lambda_{k}(H_{2n})  \\
                  & =  w_{k}^{2n} - I_{n}  R_{2n} w_{k}^{2n} \lambda_{k}(H_{n}^{-1})  \lambda_{k}(H_{2n}) \\
                  & \approx  w_{k}^{2n} -  w_{k}^{2n} \lambda_{k}(H_{n}^{-1})  \lambda_{k}(H_{2n})        \\
                  & \approx  w_{k}^{2n} (1-   \lambda_{k}(H_{n}^{-1})  \lambda_{k}(H_{2n}))
\end{align*}

(b) \\
We already analytically derived the eigenvalues for $H_{n}$. For the plot see code.\\
TG iterations may amplify smooth modes when $\rho_{k} >1$. \\
(c)\\
See code for the plots. $\rho_{k} >1$ when the sign changes. In the previous case the
index closest to the sign change is $k = 6$. No, the smoother leaves smooth modes almost unchanged.



\section{Solving the complex-valued Helmholtz problem using Multigrid}


\subsection{1D model problem}
(a) \\
see code

(b)\\
For a point source it may not be obvious but the solutions
for complex shifted problem is very similar.

(c) \\
see code

\subsection{LFA analysis of the $\omega$-Jacobi smoother}
(a) \\
Already answered in previous question. $\rho$ is still $|G(\theta)|$ almost the same reasoning.
\begin{equation}
    |G(\theta)| = |a + b \cos(\theta) + c \cos(\theta)i|
\end{equation}
with $a,b \in \mathbb{R}^{+}$ and $c \in  \mathbb{R}$ still gets optimized when $\cos\left(\theta\right)$
gets optimized. Using that argument requires that $2n^{2}+R(\sigma)\geq 0$ which follows from the criterium on
$n$ we placed at the start.

(b) \\
Depending on $\beta$ the smoother may be stable. \\
(c) \\
We think $|G(\pi)|$ or $|G(\frac{\pi}{2})|$. We have numerical evidence not a proof yet.
(d) \\
Eyeballing the plot we made $\omega \approx 0.65$ is good.

\subsection{Spectral analysis of the two-grid correction scheme}
(a) \\
Already did that. See code. Well with the formula not numerical eigenvalues ... \\
(b) \\
The instability in $\rho_{k}$ dampens.


\subsection{2D model problem}
(a) \\
Solve happens in the code.
(b) \\
Less iterations are needed for the same amount of error. This is because
$\sigma$ the source of the problem get relatively smaller to $n^{2}$ in
our convergence factors, asymptotically behavior should go back the
Laplace problem.

\subsection{Aggressive coarsening}
(a)\\
Not sure here is our guess:(TG = two grid,FG= four grid)
\begin{equation}
    FG = id -I_{2n} I_{n} H_{n}^{-1} R_{2n} R_{4n} H_{4n}.
\end{equation}
Not sure what is meant by the eigenmode analysis but what we previously
did does generalizes. The eigenmodes that are complementary with $w_{k}^{4n}$ on
the $n$ grid are: $w_{4n-k}^{4n},w_{2n-k}^{4n},w_{2n+k}^{4n}$. \\
(b)\\
Our solver diverged. Basically we are skipping a smoothing step which
makes the interpolation worse. Also relevant is that convergence
for this problem is better on larger grids. In the FG eigenmode analysis, interpolation error is worse
and eigenvalues are less similar which suggest that only the smoothest modes of the errors get dampend.
Accuracy drops but the trade off is that aggressive coarsening is less expensive about $\frac{1}{2^{d}}$
cheaper on everything but the operations on the base grid. (Half of the geometric series is gone.)

\section{Multigrid as a preconditioner for Krylov subspace methods}

\subsection{MG-GMRES for the indefinite Helmholtz problem}
(a) \\
See code.
(b) \\
Convergence is shifted down and is similar.\\
(c) \\
We tried different $\beta$. It seems that to small $\beta$ suffers from the instability problem and
to big $\beta$ would make the problem to different to use to precondition.

\end{document}

% Base your answers on the thesis.  
% You are expert in giving and revising homework.
%
% Answer as a markdown file.
% Use subnumerings in mark down form in numbering the question 
% that works in a markdown file and use different headers. Like this: 
%
% #  1. Question?
% answer
%
% Here are the questions : 
% 1. Are all requirements of the project met?
% 2. Are there any writing errors?
% 3. What are the most unclear parts in the report?
% 4. Suggest improvements. 