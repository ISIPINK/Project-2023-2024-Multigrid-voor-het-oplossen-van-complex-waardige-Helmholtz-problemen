
\documentclass[a4paper,12pt]{article}

\setlength{\textwidth}{15.0cm}
\setlength{\textheight}{24.0cm}
\setlength{\topmargin}{0cm}
\setlength{\headsep}{0cm}
\setlength{\headheight}{0cm}
\pagestyle{plain}

\usepackage{hyperref}
\hypersetup{
    colorlinks=true,
    linkcolor=blue,
    filecolor=magenta,      
    urlcolor=blue,
    citecolor=blue,
    linktoc=page
}
\usepackage[dvips]{epsfig}
\usepackage{tikz}
\usepackage[english]{babel}
\usepackage{caption}
\captionsetup{font=it}
\usepackage[autostyle, english=american]{csquotes}

\usepackage[
backend=biber,
style=alphabetic,
]{biblatex}

\renewcommand{\bibfont}{\footnotesize}

\usepackage{amsmath,amssymb,amsthm}

\usepackage{comment}
\usepackage{listings}

% \addbibresource{bibliography2.bib} 
\setlength{\parindent}{0pt}

\selectlanguage{english}
\begin{document}

\title{Multigrid for solving complex-valued Helmholtz problems}
\author{Isidoor Pinillo Esquivel}
\date{\today}
\maketitle

\section{Failure of the Multigrid method for Helmholtz problems: analysis}

\subsection{Discretization}
(a)
\begin{align*}
    10                    & \leq \lambda \text{ \#gridpoints} \Leftrightarrow                \\
    10                    & \leq \frac{2\pi}{\sqrt{|\sigma|}} \frac{1}{h^{d}}\Leftrightarrow \\
    \sqrt{|\sigma|} h^{d} & \leq \frac{2 \pi}{10} \approx 0.625.
\end{align*}

(b)
\[
    \text{\# roosterpunten} = \frac{10 \sqrt{600}}{2 \pi}
    .\]

\subsection{1D model problem}
(a) \\
To proof : $H^{2 h} \neq I_h^{2 h} H^h I_{2 h}^h$.
\begin{align*}
    H^{2h} & = H_n = A_n +\sigma {id}_n          \\
    H^{h}  & = H_{2n} = A_{2n} +\sigma {id}_{2n}
\end{align*}
Assume that $A_n =I_h^{2 h} A_{2n} I_{2 h}^h= R_{2n} A_{2n} I_n$. By linearity it is  sufficient
to proof:
\begin{align*}
    \sigma {id}_n  & \neq \sigma R_{2n} {id}_{2n} I_{n} \Leftrightarrow \\
    {id}_n         & \neq R_{2n} I_{n} \Leftarrow                       \\
    (id_n)_{00} =1 & \neq \frac{3}{4} = (R_{2n} I_{n})_{00}
\end{align*}

First equivalence follows from $\sigma \neq 0$. The assumption and the last inequality depends on the definition of
restriction and interpolation. \\
(b) \\
See code/main.ipynb for code and plots. \\
(c) \\
There exists a closed formula for eigenvalues and eigenvectors of tridiagonal toeplitz matrix. It is just tedious to
use. Alternatively the eigenvalues and eigenvectors can be derived from the Poisson problem ($\sigma=0$) because
\begin{align*}
    Av             & = \lambda v \Rightarrow \\
    (A+\sigma id)v & = Av + \sigma v         \\
                   & = (\lambda+\sigma)v
\end{align*}
i.e. eigenvectors stay the same and eigenvalues get shifted by $\sigma$. \\

(d) \\
See code/main.ipynb for the plot. $\sigma = 0$ is the boundary where $H$ goes from indefinite to definite.





\subsection{LFA analysis of the $\omega$-Jacobi smoother}

\subsection{Spectral analysis of the two-grid correction scheme}

\section{Solving the complex-valued Helmholtz problem using Multigrid}

\subsection{1D model problem}

\subsection{LFA analysis of the $\omega$-Jacobi smoother}

\subsection{Spectral analysis of the two-grid correction scheme}

\subsection{2D model problem}

\subsection{Aggressive coarsening}

\section{Multigrid as a preconditioner for Krylov subspace methods}

\subsection{MG-GMRES for the indefinite Helmholtz problem}

\end{document}

% Base your answers on the thesis.  
% You are expert in giving and revising homework.
%
% Answer as a markdown file.
% Use subnumerings in mark down form in numbering the question 
% that works in a markdown file and use different headers. Like this: 
%
% #  1. Question?
% answer
%
% Here are the questions : 
% 1. Are all requirements of the project met?
% 2. Are there any writing errors?
% 3. What are the most unclear parts in the report?
% 4. Suggest improvements. 